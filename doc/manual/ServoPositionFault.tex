\newcommand{\spfdPFD}{\textbf{Position Fault Detection}}
\newcommand{\spfdPET}{\textbf{Position Error Threshold}}
\newcommand{\spfdPED}{\textbf{Position Error Delay}}
\newcommand{\spfdPES}{\textbf{Position Error Samples}}
\newcommand{\spfdETP}{\textbf{Error Throw Period}}

\section{Servo Position Fault Detection}

\subsection{\secnameConfig}

Configuration parameters for this mechanism are located in \emph{/etc/hardware/lsct/a500.ini}.

\beginconfigtable
  \hline
  \tableheader
  \hline
  \spfdPFD{} & PFD  & false & enables or disables position fault detections. \\
  \hline
  \spfdPET{} & PET & 0.2 & threshold of the error between the actuation command and servo actual position above which a detection is made. \\
  \hline
  \spfdPED{} & PED & 5.0 & amount of time during which the position error must remain above the threshold, so that a fault detection is triggered. \\
  \hline
  \spfdPES{} & PES & 5 & number of samples used in the moving average filtering the position error. \\
  \hline
  \spfdETP{} & ETP & 20.0 & cooldown time after which a current fault error can be thrown to the output \\
  \hline
\end{tabular}

\subsection{\secnameDescription}

The position values of the servos (if available) are assumed to be synchronized with the command values, in other words, the adcs reading those positions are assumed to be properly calibrated.
Those positions are constantly being read. The error between them and the last command sent to the servo is computed.
The error value being read can be filtered with a moving average that will use \spfdPES{} samples for that purpose.
If that error goes above \spfdPET{} and remains above that value for more than \spfdPED{}, a position fault error will be thrown:
\begin{quote}
  potential fault in servo \#n, position error above X
\end{quote}
The same error will only be thrown to the output again if more than \spfdETP{} seconds have passed since the last error was thrown.

\subsection{\secnameFlowchart}

Figure~\ref{fig:servoposfault} depicts the flowchart of how the fault detection method works. The variable \emph{error} represents the absolute value of the diference between the servo's command and its actual position.

\begin{figure}[htbp]
\begin{center}
\includegraphics*[viewport= 147 380 409 670, scale=0.8]{figures/servopositionfault.pdf}
\end{center}
\caption{Servo position fault detection flowchart.}\label{fig:servoposfault}
\end{figure}
